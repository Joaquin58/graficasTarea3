\documentclass{article}

%Idioma
\usepackage[spanish]{babel}

% Símbolos
% \usepackage{recycle}
\usepackage{amsfonts}
\usepackage{amsmath}
\usepackage{amssymb}

% Figuras
\usepackage{graphicx}

% Gráficas
\usepackage{tikz}
\usetikzlibrary{calc}

% Estilo Tikzl
\tikzstyle{edge}=[shorten <=2pt, shorten >=2pt, >=stealth, line width=1.1pt]
\tikzstyle{vertex}=[circle, fill=white, draw, minimum size=5pt, inner sep=0pt,
				outer sep=0pt]
\tikzstyle{blackV}=[circle, fill=black, draw, minimum size=5pt, inner sep=0pt,
				outer sep=0pt]
\tikzstyle{arc}=[->,shorten <=3pt, shorten >=3pt, >=stealth, line width=1.1pt]
\tikzstyle{myloop}=[style={},shorten <=1pt, shorten >=1pt, >=stealth,
				line width=1.1pt, loop]

% Márgenes
\usepackage[letterpaper,headheight=20pt]{geometry}
\addtolength{\textheight}{1.5cm}

% Encabezados y Pies de Página
\usepackage{fancyhdr}
% Información del Encabezado
\lhead{Profesor: Esteban Contreras\\
Ayudantes: Luis Proudinat y Boris Belmont}
\rhead{Gráficas y Juegos\\ Grupo 4326 (2025-2)}
\renewcommand{\headrulewidth}{0pt}

%   %  % % %%%%%%%%%%%%%%%%%%%%%%%%%%%%%%%%%%%%%% % %  %   %
%   %  % % %%%%%%%%%%%%%%%%%%%%%%%%%%%%%%%%%%%%%% % %  %   %
%   %  % % %%       Gráfica en encabezado      %% % %  %   %
%   %  % % %%%%%%%%%%%%%%%%%%%%%%%%%%%%%%%%%%%%%% % %  %   %
%   %  % % %%%%%%%%%%%%%%%%%%%%%%%%%%%%%%%%%%%%%% % %  %   %
\makeatletter
\def\headrule{{\if@fancyplain\let\headrulewidth\plainheadrulewidth\fi
\hrule\@height\headrulewidth\@width\headwidth
\vspace{0.1cm}
\begin{tikzpicture}
  \begin{scope}[scale=0.6]
   \foreach \x in {0,2,...,24} {
      \node [vertex,thick] (\x) at (\x,0){};
   }
   \foreach \x in {1,3,...,25} {
      \node [blackV] (\x) at (\x,0){};
   }
   \foreach \x/\y in {0,2,...,24} {
     \pgfmathsetmacro\result{\x + 1}
     \draw [->,>=stealth,shorten <=3.5pt, shorten >=3.5pt,line width=0.7pt] (\x,0) to (\result,0);
   }
   \foreach \x/\y in {1,3,...,23} {
     \pgfmathsetmacro\result{\x + 1}
     \draw [<-,>=stealth,shorten <=3.5pt, shorten >=3.5pt,line width=0.7pt] (\x,0) to (\result,0);
   }
  \end{scope}
\end{tikzpicture}
\vskip-\headrulewidth
\vskip-1.5pt}}
\makeatother
%   %  % % %%%%%%%%%%%%%%%%%%%%%%%%%%%%%%%%%%%%%% % %  %   %
%   %  % % %%%%%%%%%%%%%%%%%%%%%%%%%%%%%%%%%%%%%% % %  %   %
%   %  % % %%%%%%%%%%%%%%%%%%%%%%%%%%%%%%%%%%%%%% % %  %   %
%   %  % % %%%%%%%%%%%%%%%%%%%%%%%%%%%%%%%%%%%%%% % %  %   %

% Estilo
\pagestyle{fancyplain}

% Macros
\newcommand{\set}[1]{\left\{ #1 \right\}}
\newcommand{\diam}{\textnormal{diam}}
\begin{document}
    \begin{enumerate}
        \section*{\LARGE{Tarea 3}}
        \subsection*{0.5 Puntos}
        % \item Dada una gráfica simple $G$, determina su número de subgráficas inducidas y su número de subgráficas inducidas por aristas.

Decimos que una gráfica inducida por $S$ $\displaystyle (G\left[S\right])$ donde $S\subseteq V_G$, tal que, es la gráfica cuyo conjunto de vértices es $S$ y en conjunto de aristas es es $\{uv\in E\big| u,v\in S\}$

En este caso, el número de subgráficas inducidas es el número de potencias de $V_G$ el cual es $2^n$ donde n es $\big|V_G\big|$.

Por lo tanto, el número de subgráficas inducidas de $G$ es $2^{|V(G)|}$.

La prueba es análoga para una subgráfica inducida por aristas.

Decimos que una subgráfica inducida por aristas en $G$ es la gráfica cuyos vértices son los vertices de G tal que son extremos por una arista que vive en $E'\subseteq E_G$

En ese caso las subgráficas posibles son todos los subconjuntos de $E_G$ que es el la cardinalidad del conjunto potencia respecto a $E_G$, esto es $\big|\mathcal{P}_{E_G}=2^n\big|$ donde $n$ es $\big|E_G\big|$
Por lo tanto, el número de subgráficas inducidas por aristas es $2^{|E(G)|}$.

        \item Sea $G$ una gráfica.
\begin{enumerate}
    \item Demuestra que, si $G$ es simple, entonces cada ciclo de longitud mínima en $G$ es una subgráfica inducida por vértices.

          Recordemos que un ciclo es un circuito que no repite vértices salvo el primero que es igual al último y una gráfica inducida es una subgráfica que contiene todos los vértices de $C$ y todas las aristas de $G$ que conectan los vértices de $C$.

          %\textbf{no necesario leer}-- (si tenemos un ciclo c que no es inducido por vertices, quiere decir que no contiene las aristas correspondientes que existen en G, basta encontrar una arista de dos vértices no consecutivos de C que existe en G pero que no existe en C, esto pasa porque G al ser  gráfica simple entonces la arista faltante no es arista múltiple o un lazo, por lo que la arista tiene que incidir a dos vertices no consecutivos en G; esta arista que vive en G haría más pequeño el ciclo ya que lo atraviesa generando un ciclo más corto, entonces el ciclo C' que si tiene esa arista faltante es G[C'] y es de longitud más prqueña que C, lo que implica que si C no es inducido entonces no es el más pequeño, lo que es equivalente a decir que si c es un ciclo más pequeño entonces es inducido por vértices.)--

          \textbf{Dem:} Por contrapuesta.

          Sea $G$ un agráfica simple y $C=(v_1,v_2,\cdots,v_n,v_1)$ un ciclo en $G$, si $C$ no es una subgráfica inducida por vértices de $G$, entonces $C$ no es de longitud mínima.

          Como $C$ no es inducida por vértices, por definición de ser gráfica inducida por vértices existe una arista $uv\in E_G$ tal que $uv\notin E_C$ para $u,v$ no consecutivos, esto pasa por definición de $G$, al ser simple la arista faltante no es arista múltiple o un lazo, por lo que la arista tiene que incidir a dos vertices no consecutivos en G.

          Tomese el siguiente ejemplo.
          \begin{figure}[!hbt]
            \begin{center}
                \begin{tikzpicture}
                    \begin{scope}[xshift=-3cm, yshift = 0cm, scale=1]
                        \foreach \i in {0,...,4}
                        \node [vertex] (\i) at ({90+(\i*72)}:1)[label={90+(\i*72)}:$v_\i$]{};
                        \foreach \i/\j in {0/1,0/2, 1/2, 2/3, 3/4, 4/0}
                        \draw [edge] (\i) to node{} (\j);
                        \node[rectangle] (n) at (1.5,0)[]{$G$};
                    \end{scope}
                    \begin{scope}[xshift=.5cm, yshift = 0cm, scale=1]
                        \foreach \i in {0,...,4}
                        \node [vertex] (\i) at ({90+(\i*72)}:1)[label={90+(\i*72)}:$v_\i$]{};

                        \foreach \i/\j in {0/1, 1/2, 2/3, 3/4, 4/0}
                        \draw [edge] (\i) to node{} (\j);
                        \node[rectangle] (n) at (1.5,0)[]{$C$};
                    \end{scope}
                    \begin{scope}[xshift=4.5cm, yshift = 0cm, scale=1]
                        \foreach \i in {0,...,2}
                        \node [vertex] (\i) at ({90+(\i*72)}:1)[label={90+(\i*72)}:$v_\i$]{};

                        \foreach \i/\j in {0/1,0/2, 1/2, 2/0}
                        \draw [edge] (\i) to node{} (\j);
                        \node[rectangle] (n) at (0.5,0)[]{$C'$};
                    \end{scope}
                    \begin{scope}[xshift=6cm, yshift = 0cm, scale=1]
                        \foreach \i in {2,3,4,0}
                        \node [vertex] (\i) at ({-120+((\i-2)*72)}:1)[label={90+(\i*72)}:$v_\i$]{};

                        \foreach \i/\j in {2/3,3/4, 4/0, 0/2}
                        \draw [edge] (\i) to node{} (\j);
                        \node[rectangle] (n) at (1.5,0)[]{$C''$};
                    \end{scope}
                \end{tikzpicture}
            \end{center}
        \end{figure}

          Entonces se puede divir a $G$ en dos ciclos, en particular el ciclo $C'=(v_0,v_1,v_2,v_0)$ que está formado por la arista $v_0v_2$ que vive en $G$ y que no vive en $C$, lo que la combierte en un ciclo de longitud más corto que la longitud de $C$. Esto implica que si el ciclo $C$ no es inducido entonces no es de longitud mínima, lo que es equivalente a decir que si $C$ es un ciclo más pequeño entonces es inducido por vértices.
          
    \item Prueba que lo anterior no es necesariamente cierto si $G$ no es simple.
          Daremos un contrajemplo.

          Cosiderece la siguiente gráfica
          \begin{figure}[!hbt]
              \begin{center}
                  \begin{tikzpicture}

                      \begin{scope}[scale=2, xshift=0cm, yshift = 0cm]
                        %   \node [vertex] (1) at (210:2)[label=210:$v_4$]{};
                          \node [vertex] (1) at (0,-1)[label=270:$v_1$]{};

                        %   \draw [edge,bend right=30] (3) to node[label=270:$e_3$]{} (1);
                        %   \draw [edge, bend left=50] (3) to node[label=90:$e_5$]{} (1);
                          \draw [myloop,in=30,out=-30,looseness=45] (1) to node[label=10:$e_{1}$]{} (1);
                          \draw [myloop,in=150,out=210,looseness=45] (1) to node[label=100:$e_{2}$]{} (1);
                      \end{scope}
                  \end{tikzpicture}
              \end{center}
              \caption{contrajemplo.}
          \end{figure}

          Notemos el ciclo $C=(v_1,e_6,v_1)$ de longitud minima 1 y para que $C$ sea una subgráfica inducida por vertices de G debe ocurrir que $E_C$ contenga todas las aristas de $G$ que une a los vertices en $V_C$ pero vemos que esto no pasa porque el lazo $e_7\notin E_C$ pero $e_7\in E_G$ por lo que no es una subgráfica inducida por vértices.
\end{enumerate}\newpage
        % \subsection*{1 Punto}
        % \item \label{ej:Petersen} \begin{enumerate}
    \item Encuentra una gráfica simple $Q$ de orden 10, tamaño 15, regular, que no
    contenga ciclos inducidos de orden 3 ni 4.
    \item Demuestra que la gráfica $Q$ del inciso anterior es isomorfa a la
    gráfica de la Figura \ref{fig:Petersen}.
  \end{enumerate}
  \begin{figure}[!hbt]
	\begin{center}
		\begin{tikzpicture}

      \begin{scope}[scale=1]
				\foreach \i in {0,...,4}
					\node [vertex] (\i) at ({90+(\i*72)}:2)[label={35+(\i*72)}:$v_\i$]{};
				\foreach \i in {5,...,9}
					\node [vertex] (\i) at ({90+(\i*72)}:1)[label={35+(\i*72)}:$v_\i$]{};

        \foreach \i in {0,...,4} {
					\draw let \n1={int(mod(\i+1,5))} in [edge] (\i) to (\n1);
					\draw let \n1={int(\i+5)} in [edge] (\i) to (\n1);
        }
					
        \foreach \i in {5,6,7} {
					\draw let \n1={int(\i+2)} in [edge] (\i) to (\n1);
        }
					
				\foreach \i/\j in {9/6,8/5}
					\draw [edge] (\i) to node{} (\j);

			\end{scope}

		\end{tikzpicture}
	\end{center}
	\caption{La gráfica del ejercicio \ref{ej:Petersen}.}
	\label{fig:Petersen}
\end{figure}
  \newpage
        % \item \begin{enumerate}
    \item Demuestra que toda gráfica simple de orden $n \ge 6$ contiene como
    subgráfica o bien a $K_3$ o bien a $O_3$.
    
    \item Encuentra todas las gráficas simples $G$ tales que tanto $G$ como
    $\overline{G}$ son bipartitas. Justifica tu respuesta.
    
  \end{enumerate}
        % \item Prueba que si $G$ es una gráfica simple, conexa y no completa, entonces
$G$ contiene a $P_3$ como subgráfica inducida.


        % \item Sean $k$ un entero, $k \ge 2$, y $G$ una gráfica simple. Demuestra que $G$
es $k$-partita completa si y sólo si $G$ es libre de $\{K _{k+1}, \overline{P_3}\}$.

Recordemos la definción de k-partita, esto es, dado un entero positivo $k$, decimos que una gráfica simple $G$ es \textit{k-partita completa} si su conjunto de vértices admite una partición \\$(V_1, V_2,\dots, V_k)$ donde cada $V_i$ es un conjunto independiente, y $V_i$ es completamente adyacente a $V_j$ siempre que $i \neq j$.

Recordemos que $G$ es libre de $\{K _{k+1}, \overline{P_3}\}$ cuando $G$ no contiene a $K _{k+1}$
        % \item Sea $G$ una gráfica con $V_G = \{v_1, v_2, \dots ,v_n\}$, y sea $A$ su
matriz de adyacencia asociada a dicho ordenamiento de $V_G$. Demuestra que, para
cada entero no negativo $k$, el número de $v_iv_j$-caminos de longitud $k$ es
$(A^k)_{i,j}$
    \end{enumerate}
\end{document}