\label{ej:Petersen} \begin{enumerate}
    \item Encuentra una gráfica simple $Q$ de orden 10, tamaño 15, regular, que no
    contenga ciclos inducidos de orden 3 ni 4.
    \item Demuestra que la gráfica $Q$ del inciso anterior es isomorfa a la
    gráfica de la Figura \ref{fig:Petersen}.
  \end{enumerate}
  \begin{figure}[!hbt]
	\begin{center}
		\begin{tikzpicture}

      \begin{scope}[scale=1]
				\foreach \i in {0,...,4}
					\node [vertex] (\i) at ({90+(\i*72)}:2)[label={35+(\i*72)}:$v_\i$]{};
				\foreach \i in {5,...,9}
					\node [vertex] (\i) at ({90+(\i*72)}:1)[label={35+(\i*72)}:$v_\i$]{};

        \foreach \i in {0,...,4} {
					\draw let \n1={int(mod(\i+1,5))} in [edge] (\i) to (\n1);
					\draw let \n1={int(\i+5)} in [edge] (\i) to (\n1);
        }
					
        \foreach \i in {5,6,7} {
					\draw let \n1={int(\i+2)} in [edge] (\i) to (\n1);
        }
					
				\foreach \i/\j in {9/6,8/5}
					\draw [edge] (\i) to node{} (\j);

			\end{scope}

		\end{tikzpicture}
	\end{center}
	\caption{La gráfica del ejercicio \ref{ej:Petersen}.}
	\label{fig:Petersen}
\end{figure}
  