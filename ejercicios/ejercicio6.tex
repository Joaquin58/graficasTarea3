Sean $k$ un entero, $k \ge 2$, y $G$ una gráfica simple. Demuestra que $G$
es $k$-partita completa si y sólo si $G$ es libre de $\{K _{k+1}, \overline{P_3}\}$.

Recordemos la definción de k-partita, esto es, dado un entero positivo $k$, decimos que una gráfica simple $G$ es \textit{k-partita completa} si su conjunto de vértices admite una partición \\$(V_1, V_2,\dots, V_k)$ donde cada $V_i$ es un conjunto independiente, y $V_i$ es completamente adyacente a $V_j$ siempre que $i \neq j$.

Recordemos que $G$ es libre de $\{K _{k+1}, \overline{P_3}\}$ cuando $G$ no contiene a $K _{k+1}$