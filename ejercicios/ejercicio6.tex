Sean $k$ un entero, $k \ge 2$, y $G$ una gráfica simple. Demuestra que $G$ es $k$-partita completa si y sólo si $G$ es libre de $\{K _{k+1}, \overline{P_3}\}$.

Recordemos la definción de k-partita, esto es, dado un entero positivo $k$, decimos que una gráfica simple $G$ es \textit{k-partita completa} si su conjunto de vértices admite una partición \\$(V_1, V_2,\dots, V_k)$ donde cada $V_i$ es un conjunto independiente, y $V_i$ es completamente adyacente a $V_j$ siempre que $i \neq j$.

Recordemos que $K_{k+1}$ es la gráfica completa de orden $k+1$ vértices, $\overline{P_3}$ es 
Si $G$ es \textit{k-partita completa} entonces, $G$ es libre de $\{K_{k+1},\overline{P_3}\}$

Decimos que G es libre de H si no existe algún conjunto de vertices de G,

Como $G$ es \textit{k-partita completa} entonces $V_G$ se puede dividir en subconjutos $(V_1, V_2,\dots, V_k)$ independientes.
Por el principio del palomar, si $G$ contiene un $K_{k+1}$ tal que su orden es de $k+1$ vérticesm, entonces, al menos dos vértices deben caer en la misma partición.