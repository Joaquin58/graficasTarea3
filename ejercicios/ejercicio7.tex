Sea $G$ una gráfica con $V_G = \{v_1, v_2, \dots ,v_n\}$, y sea $A$ su matriz de adyacencia asociada a dicho ordenamiento de $V_G$. Demuestra que, para cada entero no negativo $k$, el número de $v_iv_j-caminos$ de longitud $k$ es $(A^k)_{i,j}$

Sea $A=(a_{ij})\in M_{n\times n}{(\mathbb{R})}$ una matriz de adyacencia asociada al ordenamiento de la gráfica G.

Lo demostraremos por inducción.

\textbf{Casos Base:} $k=1$, que es la longitud del $v_iv_j-camino$, este existe si y solo si $v_iv_j\in E_G$ pero esto está dado por la matriz de adyacencia asociada a la gráfica $G$ que es $(A)_{i,j}$ por definción.

\textbf{Hipótesis de inducción:} Nosostros afirmamos que para todo entero $k=n$ no negativo es verdad que el número de $v_iv_j-camino$ de longitud $n$ es $(A^k)_{i,j}$.

\textbf{Paso inductivo:} Por demostrar que se cumple para $n+1$.
Probemos que el número de $v_iv_j-camino$ de longitud $n+1$ es $(A^k)_{i,j}$

Notemos que $A^{n+1}=A\cdot A^n$ , por hipótesis de inducción el número de caminos de $n$ desde $v_i$ a $v_l$  es $\displaystyle(A^{n}\cdot A) =\sum_{l=1}^{m} (A^{n})_{i,l} \cdot (A)_{l,j}$.

donde $(A^{n})_{i,l}$ cuenta de los caminos de longitud $n$ desde $v_i$ hasta $v_l$ y donde $(A)_{l,j}$ es el número de caminos que existen entre $v_l$ y $v_j$ y aporta en almenos 1 valor a la longitud del camino, dicho de otro modo indica la posibilidad de moverse de $v_l$ a $v_j$ en al menos un paso, i.e $(v_0,v_1,\dots,v_l,v_j)$. 

Por lo tanto el número de caminos de longitud $k+1$ desde $v_{i}$ hasta $v_j$ es $(A^k)_{i,j}$.