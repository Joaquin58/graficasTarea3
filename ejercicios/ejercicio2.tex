Sea $G$ una gráfica.
\begin{enumerate}
    \item Demuestra que, si $G$ es simple, entonces cada ciclo de longitud mínima en $G$ es una subgráfica inducida por vértices.

          Recordemos que un ciclo es un circuito que no repite vértices salvo el primero que es igual al último y una gráfica inducida es una subgráfica que contiene todos los vértices de $C$ y todas las aristas de $G$ que conectan los vértices de $C$.

          %\textbf{no necesario leer}-- (si tenemos un ciclo c que no es inducido por vertices, quiere decir que no contiene las aristas correspondientes que existen en G, basta encontrar una arista de dos vértices no consecutivos de C que existe en G pero que no existe en C, esto pasa porque G al ser  gráfica simple entonces la arista faltante no es arista múltiple o un lazo, por lo que la arista tiene que incidir a dos vertices no consecutivos en G; esta arista que vive en G haría más pequeño el ciclo ya que lo atraviesa generando un ciclo más corto, entonces el ciclo C' que si tiene esa arista faltante es G[C'] y es de longitud más prqueña que C, lo que implica que si C no es inducido entonces no es el más pequeño, lo que es equivalente a decir que si c es un ciclo más pequeño entonces es inducido por vértices.)--

          \textbf{Dem:} Por contrapuesta.

          Sea $G$ un agráfica simple y $C=(v_1,v_2,\cdots,v_n,v_1)$ un ciclo en $G$, si $C$ no es una subgráfica inducida por vértices de $G$, entonces $C$ no es de longitud mínima.

          Como $C$ no es inducida por vértices, por definición de ser gráfica inducida por vértices existe una arista $uv\in E_G$ tal que $uv\notin E_C$ para $u,v$ no consecutivos, esto pasa por definición de $G$, al ser simple la arista faltante no es arista múltiple o un lazo, por lo que la arista tiene que incidir a dos vertices no consecutivos en G.

          Tomese el siguiente ejemplo.
          \begin{figure}[!hbt]
            \begin{center}
                \begin{tikzpicture}
                    \begin{scope}[xshift=-3cm, yshift = 0cm, scale=1]
                        \foreach \i in {0,...,4}
                        \node [vertex] (\i) at ({90+(\i*72)}:1)[label={90+(\i*72)}:$v_\i$]{};
                        \foreach \i/\j in {0/1,0/2, 1/2, 2/3, 3/4, 4/0}
                        \draw [edge] (\i) to node{} (\j);
                        \node[rectangle] (n) at (1.5,0)[]{$G$};
                    \end{scope}
                    \begin{scope}[xshift=.5cm, yshift = 0cm, scale=1]
                        \foreach \i in {0,...,4}
                        \node [vertex] (\i) at ({90+(\i*72)}:1)[label={90+(\i*72)}:$v_\i$]{};

                        \foreach \i/\j in {0/1, 1/2, 2/3, 3/4, 4/0}
                        \draw [edge] (\i) to node{} (\j);
                        \node[rectangle] (n) at (1.5,0)[]{$C$};
                    \end{scope}
                    \begin{scope}[xshift=4.5cm, yshift = 0cm, scale=1]
                        \foreach \i in {0,...,2}
                        \node [vertex] (\i) at ({90+(\i*72)}:1)[label={90+(\i*72)}:$v_\i$]{};

                        \foreach \i/\j in {0/1,0/2, 1/2, 2/0}
                        \draw [edge] (\i) to node{} (\j);
                        \node[rectangle] (n) at (0.5,0)[]{$C'$};
                    \end{scope}
                    \begin{scope}[xshift=6cm, yshift = 0cm, scale=1]
                        \foreach \i in {2,3,4,0}
                        \node [vertex] (\i) at ({-120+((\i-2)*72)}:1)[label={90+(\i*72)}:$v_\i$]{};

                        \foreach \i/\j in {2/3,3/4, 4/0, 0/2}
                        \draw [edge] (\i) to node{} (\j);
                        \node[rectangle] (n) at (1.5,0)[]{$C''$};
                    \end{scope}
                \end{tikzpicture}
            \end{center}
        \end{figure}

          Entonces se puede divir a $G$ en dos ciclos, en particular el ciclo $C'=(v_0,v_1,v_2,v_0)$ que está formado por la arista $v_0v_2$ que vive en $G$ y que no vive en $C$, lo que la combierte en un ciclo de longitud más corto que la longitud de $C$. Esto implica que si el ciclo $C$ no es inducido entonces no es de longitud mínima, lo que es equivalente a decir que si $C$ es un ciclo más pequeño entonces es inducido por vértices.
          
    \item Prueba que lo anterior no es necesariamente cierto si $G$ no es simple.
          Daremos un contrajemplo.

          Cosiderece la siguiente gráfica
          \begin{figure}[!hbt]
              \begin{center}
                  \begin{tikzpicture}

                      \begin{scope}[scale=2, xshift=0cm, yshift = 0cm]
                        %   \node [vertex] (1) at (210:2)[label=210:$v_4$]{};
                          \node [vertex] (1) at (0,-1)[label=270:$v_1$]{};

                        %   \draw [edge,bend right=30] (3) to node[label=270:$e_3$]{} (1);
                        %   \draw [edge, bend left=50] (3) to node[label=90:$e_5$]{} (1);
                          \draw [myloop,in=30,out=-30,looseness=45] (1) to node[label=10:$e_{1}$]{} (1);
                          \draw [myloop,in=150,out=210,looseness=45] (1) to node[label=100:$e_{2}$]{} (1);
                      \end{scope}
                  \end{tikzpicture}
              \end{center}
              \caption{contrajemplo.}
          \end{figure}

          Notemos el ciclo $C=(v_1,e_6,v_1)$ de longitud minima 1 y para que $C$ sea una subgráfica inducida por vertices de G debe ocurrir que $E_C$ contenga todas las aristas de $G$ que une a los vertices en $V_C$ pero vemos que esto no pasa porque el lazo $e_7\notin E_C$ pero $e_7\in E_G$ por lo que no es una subgráfica inducida por vértices.
\end{enumerate}